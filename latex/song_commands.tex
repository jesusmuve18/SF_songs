% Paquete para realizar la función módulo
\usepackage{pgfmath}

% Paquete para figuras
\usepackage{tikz}

% Paquete para comparar cadenas con caracteres especiales
\usepackage{xstring}

% Paquete para eliminar sangrías en el primer párrafo
\usepackage{indentfirst}

% Paquete para poder escribir encima de las palabras
\usepackage{amsmath}

% Paquete para cambiar el color del texto
\usepackage{xcolor}

% Paquete para operaciones aritméticas con variables
\usepackage{calc}

% Paquete para disponer el texto en dos columnas
\usepackage{multicol}

% Paquete para ajustar el interlineado
\usepackage{setspace} 

% Paquete para referencias
\usepackage{hyperref}

% Paquete para parámetros adicionales
\usepackage{pgfkeys}

% Paquete para la imagen de la portada
\usepackage{graphicx}

% Paquete para el correcto posicionamiento de los diagramas
\usepackage{float}

% Paquete para quitar las etiquetas de los diagramas
\usepackage{caption}

% Tipografía
\usepackage[defaultfam,tabular,lining]{montserrat} %% Option 'defaultfam'
%% only if the base font of the document is to be sans serif
\usepackage[T1]{fontenc}
\renewcommand*\oldstylenums[1]{{\fontfamily{Montserrat-TOsF}\selectfont #1}}

% Establecer longitud de la sangría a cero
\setlength{\parindent}{0pt}

% Notación por defecto
\newcommand{\notation}{europe}

% Directorio de imágenes
\graphicspath{/home/jesusmuve/Documentos/VSCode/LaTeX/documentos/songs/}
\def\dirimages{/home/jesusmuve/Documentos/VSCode/LaTeX/documentos/songs/images/}

% Comparar cadenas con caracteres compuestos
% Sintáxis: \ifstrequal{string1}{string2}{code if are equals}{code if not}
\renewcommand{\ifstrequal}[4]{%
    \IfSubStr{#1}{\#}{%
        \IfSubStr{#2}{\#}{%
            \StrGobbleRight{#1}{1}[\strippedA]%
            \StrGobbleRight{#2}{1}[\strippedB]%
            \IfStrEq{\strippedA}{\strippedB}{%
                #3%
            }{%
                #4%
            }%
        }{%
            #4%
        }%
    }{%
        \IfSubStr{#2}{\#}{%
            #4%
        }{%
            \IfStrEq{#1}{#2}{%
                #3%
            }{%
                #4%
            }%
        }%
    }%
}

% Compara si dos valores numéricos son iguales
% Sintáxis: \ifnumequal{num1}{num2}{code if are equals}{code if not}
\renewcommand{\ifnumequal}[4]{%
    \ifnum#1=#2%
        #3%
    \else   
        #4%
    \fi
}

% Contador para la conversión de nota a valor
\newcounter{notaavalor}

% Comando que ajusta el valor del contador notaavalor según una nota introducida
\newcommand{\notaavalor}[1]{%
    \ifstrequal{#1}{C}{\setcounter{notaavalor}{0}}{%
        \ifstrequal{#1}{C\#}{\setcounter{notaavalor}{1}}{%
            \ifstrequal{#1}{D}{\setcounter{notaavalor}{2}}{%
                \ifstrequal{#1}{D\#}{\setcounter{notaavalor}{3}}{%
                    \ifstrequal{#1}{E}{\setcounter{notaavalor}{4}}{%
                        \ifstrequal{#1}{F}{\setcounter{notaavalor}{5}}{%
                            \ifstrequal{#1}{F\#}{\setcounter{notaavalor}{6}}{%
                                \ifstrequal{#1}{G}{\setcounter{notaavalor}{7}}{%
                                    \ifstrequal{#1}{G\#}{\setcounter{notaavalor}{8}}{%
                                        \ifstrequal{#1}{A}{\setcounter{notaavalor}{9}}{%
                                            \ifstrequal{#1}{A\#}{\setcounter{notaavalor}{10}}{%
                                                \ifstrequal{#1}{B}{\setcounter{notaavalor}{11}}{%
                                                    \ifstrequal{#1}{Do}{\setcounter{notaavalor}{0}}{%
                                                        \ifstrequal{#1}{Do\#}{\setcounter{notaavalor}{1}}{%
                                                            \ifstrequal{#1}{Re}{\setcounter{notaavalor}{2}}{%
                                                                \ifstrequal{#1}{Re\#}{\setcounter{notaavalor}{3}}{%
                                                                    \ifstrequal{#1}{Mi}{\setcounter{notaavalor}{4}}{%
                                                                        \ifstrequal{#1}{Fa}{\setcounter{notaavalor}{5}}{%
                                                                            \ifstrequal{#1}{Fa\#}{\setcounter{notaavalor}{6}}{%
                                                                                \ifstrequal{#1}{Sol}{\setcounter{notaavalor}{7}}{%
                                                                                    \ifstrequal{#1}{Sol\#}{\setcounter{notaavalor}{8}}{%
                                                                                        \ifstrequal{#1}{La}{\setcounter{notaavalor}{9}}{%
                                                                                            \ifstrequal{#1}{La\#}{\setcounter{notaavalor}{10}}{%
                                                                                                \ifstrequal{#1}{Si}{\setcounter{notaavalor}{11}}{}%
                                                                                            }%
                                                                                        }%
                                                                                    }%
                                                                                }%
                                                                            }%
                                                                        }%
                                                                    }%
                                                                }%
                                                            }%
                                                        }%
                                                    }%
                                                }%
                                            }%
                                        }%
                                    }%
                                }%
                            }%
                        }%
                    }%
                }%
            }%
        }%
    }%
}

% Comando que transforma un valor numérico en su nota correspondiente
\newcommand{\valoranota}[1]{%
    \ifstrequal{\notation}{europe}{%
        \ifnumequal{#1}{0}{Do}{%
            \ifnumequal{#1}{1}{Do\#}{%
                \ifnumequal{#1}{2}{Re}{%
                    \ifnumequal{#1}{3}{Re\#}{%
                        \ifnumequal{#1}{4}{Mi}{%
                            \ifnumequal{#1}{5}{Fa}{%
                                \ifnumequal{#1}{6}{Fa\#}{%
                                    \ifnumequal{#1}{7}{Sol}{%
                                        \ifnumequal{#1}{8}{Sol\#}{%
                                            \ifnumequal{#1}{9}{La}{%
                                                \ifnumequal{#1}{10}{La\#}{%
                                                    \ifnumequal{#1}{11}{Si}{-}%
                                                }%
                                            }%
                                        }%
                                    }%
                                }%
                            }%
                        }%
                    }%
                }%
            }%
        }%
    }{%
        \ifnumequal{#1}{0}{C}{%
            \ifnumequal{#1}{1}{C\#}{%
                \ifnumequal{#1}{2}{D}{%
                    \ifnumequal{#1}{3}{D\#}{%
                        \ifnumequal{#1}{4}{E}{%
                            \ifnumequal{#1}{5}{F}{%
                                \ifnumequal{#1}{6}{F\#}{%
                                    \ifnumequal{#1}{7}{G}{%
                                        \ifnumequal{#1}{8}{G\#}{%
                                            \ifnumequal{#1}{9}{A}{%
                                                \ifnumequal{#1}{10}{A\#}{%
                                                    \ifnumequal{#1}{11}{B}{-}%
                                                }%
                                            }%
                                        }%
                                    }%
                                }%
                            }%
                        }%
                    }%
                }%
            }%
        }%
    }%
}%


% Contadores necesarios para la función acorde
\newcounter{acumulador}
\newcounter{traspose}
\newcounter{cejilla}

% Función que ajusta el tono
\newcommand{\traspose}[1]{%
    \setcounter{traspose}{#1}%
}

% Función que ajusta la cejilla
\newcommand{\capo}[1]{%
    \setcounter{cejilla}{#1}%
}

% Función que dado un acorde lo ajusta según el tono
\newcommand{\acorde}[1]{%
    \notaavalor{#1}%
    \setcounter{acumulador}{\value{notaavalor}+\value{traspose}-\value{cejilla}}%
    %\pgfmathsetmacro{\modulo}{int(mod(\the\value{acumulador}, 12))}%
    \pgfmathsetmacro{\modulo}{ifthenelse(\value{acumulador}<0, int(mod(\the\value{acumulador}+12, 12)), int(mod(\the\value{acumulador}, 12)))}%
    \valoranota{\modulo}%
}

% Formato para los acordes
\newcommand{\chordstyle}[1]{%
    \textcolor{chordcolor}{\selectfont\bfseries #1}%
}

% Color para los acordes
\definecolor{chordcolor}{RGB}{255,87,51}

% Funcion que permite escribir el acorde encima de una palabra (de la letra de la canción)
% Sintáxis: \chord[m|M|7|...]{Nota}{Palabra sobre la que va la nota o acorde}
\newcommand{\chord}[3]{%
    %$\overset{\textstyle\text{\chordstyle{\acorde{#1}#2}}}{\vphantom{\text{g}} \textstyle\text{#3}}$%
    %$\overset{\raisebox{3pt}{\text{\chordstyle{\acorde{#1}#2}\vphantom{é}}}}{\textstyle\text{#3\vphantom{é}}}$% Ajusta la distancia con raisebox
    \renewcommand\useanchorwidth{T}%
    \stackon{\text{#3\vphantom{é}}}{\text{\chordstyle{\acorde{#1}#2}}}%
}

%Título
% 1. Título
% 2. Artista
% 3. Tono
% 4. Cejilla
% 5. Trasponer
\newcommand{\Song}[5]{%
    \newcommand{\originalkey}{#3}%
    \capo{#4}%
    \traspose{#5}
    %--------------------
    \begin{spacing}{1.5}
        \textbf{\Huge#1}\\ % Título
        \textbf{\chordstyle{#2}}\\ %Subtítulo
    \end{spacing}
    \vspace*{0.5cm}
    \begin{spacing}{1}
        {\setstretch{1}\textbf{Tono original:} \chordstyle{%
            \notaavalor{#3}%
            \valoranota{\value{notaavalor}}%
        }}\\ % Tono original
        \setcounter{acumulador}{\value{cejilla}}%
        \ifnumequal{\value{acumulador}}{0}{}{%
            {\textbf{Tono actual:} \chordstyle{\acorde{#3}}}\\%
            \setcounter{acumulador}{\value{traspose}}%
            {\textbf{Traspuesta:}  \ifnumequal{\the\value{acumulador}}{0}{no}{\the\value{acumulador}}}\\%
            }%
        %{\textbf{Tono de la canción:} \textcolor{chordcolor}{\acorde{#3}}}\\ %Tono de la canción
        {\textbf{Cejilla:} \ifnumequal{#4}{0}{no}{Traste #4}}\\ %Cejilla
    \end{spacing}
}

% Poner el estribillo en negrita
\newenvironment{chorus}{\bfseries \jump}{\jump}

\renewcommand{\chordstyle}[1]{%
    \textcolor{chordcolor}{\fontsize{10}{14}\bfseries\selectfont #1}%
}


% Acordes para huecos en blanco
\newcommand{\gapchord}[2]{%
    \chord{#1}{#2}{\phantom{t}}%
}

% Bis
\newcommand{\bis}{\textbf{(bis)}}

% Entorno canción
\newenvironment{song}[5]{%
    % Cambio de interlineado
    \setstretch{2.3}%
    
    \phantomsection % Crea un marcador de posición para el enlace
    \addcontentsline{toc}{subsection}{#1 (#2)} \label{sec:#1}

    \begin{flushleft}%

    \Song{#1}{#2}{#3}{#4}{#5} % Título - Artista - Tono Original - Cejilla - Trasponer
    \begin{multicols*}{2}
}{%
    \end{multicols*}
    \end{flushleft}
}

\newcommand{\jump}{\vspace*{0.5cm}}

% Parámetros por defecto para diagrama
\pgfkeys{
    /diagrama/.is family, /diagrama,
    trasteinicial/.initial=1,
    nombre/.initial=Name,
    variacion/.initial=,
    numerotrastes/.initial=3,
    cejilla/.initial=0,
    iniciocejilla/.initial=6,
    finalcejilla/.initial=1,
    height/.initial=3,
}

% Diagrama de acordes en horizontal
\newcommand{\hdiagram}[7][]{

    % Procesar las claves
    \pgfkeys{/diagrama, #1}%
    
    \begin{figure}[H]%

        % Hiperlinks
        \captionsetup{labelformat=empty} % Elimina la etiqueta de la leyenda
        \caption{}%
        \def\nombre{\pgfkeysvalueof{/diagrama/nombre}\pgfkeysvalueof{/diagrama/variacion}}%
        \label{fig:\nombre}%

        \centering

        % Ajuste tamaño figura
        \resizebox{!}{\pgfkeysvalueof{/diagrama/height}cm}{%  
            \def\ancho{3.75+\numerotrastes*1.75} % 9.75 - 8.25
            \def\nombre{\pgfkeysvalueof{/diagrama/nombre}}%
            \def\variacion{\pgfkeysvalueof{/diagrama/variacion}}%
            \def\primera{#2}%
            \def\segunda{#3}%
            \def\tercera{#4}%
            \def\cuarta{#5}%
            \def\quinta{#6}%
            \def\sexta{#7}%
            \def\cejilla{\pgfkeysvalueof{/diagrama/cejilla}}%
            \def\iniciocejilla{\pgfkeysvalueof{/diagrama/iniciocejilla}}%
            \def\finalcejilla{\pgfkeysvalueof{/diagrama/finalcejilla}}%
            \def\numerotrastes{\pgfkeysvalueof{/diagrama/numerotrastes}}%
            \def\trasteinicial{\pgfkeysvalueof{/diagrama/trasteinicial}}%

            %Grosor de líneas
            \def\lineas{0.6pt}%

            \begin{tikzpicture}%
                % %Título
                \pgfmathsetmacro{\pos}{\ancho+3.75}%
                \node at (\ancho*0.5,14.5) {\Huge \bfseries \textcolor{chordcolor}{%(6.375,14.5)
                    \notaavalor{\nombre}%
                    \valoranota{\value{notaavalor}}%
                    \variacion%
                    }%
                };%

                % Rectángulo externo
                \draw[line width=\lineas] (3.75,13.5) rectangle (\ancho,9.75);%

                % Líneas horizontales
                \draw[line width=\lineas] (3.75,12.75) -- (\ancho,12.75);%
                \draw[line width=\lineas] (3.75,12) -- (\ancho,12);%
                \draw[line width=\lineas] (3.75,11.25) -- (\ancho,11.25);%
                \draw[line width=\lineas] (3.75,10.5) -- (\ancho,10.5);%

                % Líneas verticales
                \foreach \d in {1,2,...,\numerotrastes}{%
                    \draw[line width=\lineas] (3.75+\d*1.75,13.5) -- (3.75+\d*1.75,9.75);%
                    \pgfmathsetmacro{\suma}{int(\trasteinicial + \d - 1)}%
                    \node at (2.75+1.75*\d,9.25) {\Large \suma};%
                }%

                % Círculos

                % Primera cuerda
                \ifstrequal{\primera}{0}{%
                    \draw[line width=\lineas] (3.3,13.5) circle (4pt);%
                }{%
                    \ifstrequal{\primera}{x}{%
                        \node at (3.3,13.5) {\Huge {\texttimes}};%
                    }{%
                    \pgfmathsetmacro{\pos}{int(\primera-\trasteinicial+1)}%
                        \filldraw[line width=\lineas] (2.75+1.75*\pos,13.5) circle (6pt);%
                    }%
                }%

                % Segunda cuerda
                \ifstrequal{\segunda}{0}{%
                    \draw[line width=\lineas] (3.3,12.75) circle (4pt);%
                }{%
                    \ifstrequal{\segunda}{x}{%
                        \node at (3.3,12.75) {\Huge {\texttimes}};%
                    }{%
                    \pgfmathsetmacro{\pos}{int(\segunda-\trasteinicial+1)}%
                        \filldraw[line width=\lineas] (2.75+1.75*\pos,12.75) circle (6pt);%
                    }%
                }%

                % Tercera cuerda
                \ifstrequal{\tercera}{0}{%
                    \draw[line width=\lineas] (3.3,12) circle (4pt);%
                }{%
                    \ifstrequal{\tercera}{x}{
                        \node at (3.3,12) {\Huge {\texttimes}};%
                    }{%
                        \pgfmathsetmacro{\pos}{int(\tercera-\trasteinicial+1)}%
                        \filldraw[line width=\lineas] (2.75+1.75*\pos,12) circle (6pt);%
                    }%
                }%

                % Cuarta cuerda
                \ifstrequal{\cuarta}{0}{%
                    \draw[line width=\lineas] (3.3,11.25) circle (4pt);%
                }{%
                    \ifstrequal{\cuarta}{x}{%
                        \node at (3.3,11.25) {\Huge {\texttimes}};%
                    }{%
                        \pgfmathsetmacro{\pos}{int(\cuarta-\trasteinicial+1)}%
                        \filldraw[line width=\lineas] (2.75+1.75*\pos,11.25) circle (6pt);%
                    }%
                }%

                % Quinta cuerda
                \ifstrequal{\quinta}{0}{%
                    \draw[line width=\lineas] (3.3,10.5) circle (4pt);%
                }{%
                    \ifstrequal{\quinta}{x}{%
                        \node at (3.3,10.5) {\Huge {\texttimes}};%
                    }{%
                        \pgfmathsetmacro{\pos}{int(\quinta-\trasteinicial+1)}%
                        \filldraw[line width=\lineas] (2.75+1.75*\pos,10.5) circle (6pt);%
                    }%
                }%

                % Sexta cuerda
                \ifstrequal{\sexta}{0}{%
                    \draw[line width=\lineas] (3.3,9.75) circle (4pt);%
                }{%
                    \ifstrequal{\sexta}{x}{%
                        \node at (3.3,9.75) {\Huge {\texttimes}};%
                    }{%
                    \pgfmathsetmacro{\pos}{int(\sexta-\trasteinicial+1)}%
                        \filldraw[line width=\lineas] (2.75+1.75*\pos,9.75) circle (6pt);%
                    }%
                }%
                
                % Cejilla
                \ifstrequal{\cejilla}{0}{}{%
                    \pgfmathsetmacro{\pos}{int(\cejilla-\trasteinicial+1)}%
                    \filldraw[fill=black, rounded corners=6] (2.75+1.75*\pos+0.22,-0.75*\finalcejilla+14.45) rectangle (2.75+1.75*\pos-0.22,-0.75*\iniciocejilla+14.05);
                }%
                

                % Cejuela
                \ifstrequal{\trasteinicial}{1}{%
                    \filldraw[fill=white, rounded corners=3] (3.875,13.55) rectangle (3.625,9.7);%
                }{}%
                
            \end{tikzpicture}%
        }
        
    \end{figure}
}

% Diagrama de acordes en vertical
\newcommand{\vdiagram}[7][]{

    % Procesar las claves
    \pgfkeys{/diagrama, #1}%
    \def\anchoaux{\pgfkeysvalueof{/diagrama/height}}
    \pgfmathsetmacro{\ancho}{\anchoaux*1.16}%

    \begin{figure}[H]
        % Hiperlinks
        \captionsetup{labelformat=empty} % Elimina la etiqueta de la leyenda
        \caption{}
        \def\nombre{\pgfkeysvalueof{/diagrama/nombre}\pgfkeysvalueof{/diagrama/variacion}}
        \label{fig:\nombre}

        \centering

        % Ajuste tamaño figura
        \resizebox{!}{\ancho cm}{%
            \def\ancho{3.75+\numerotrastes*1.75} % 9.75 - 8.25
            \def\nombre{\pgfkeysvalueof{/diagrama/nombre}}
            \def\variacion{\pgfkeysvalueof{/diagrama/variacion}}
            \def\trasteinicial{\pgfkeysvalueof{/diagrama/trasteinicial}}
            \def\primera{#2}
            \def\segunda{#3}
            \def\tercera{#4}
            \def\cuarta{#5}
            \def\quinta{#6}
            \def\sexta{#7}
            \def\cejilla{\pgfkeysvalueof{/diagrama/cejilla}}
            \def\iniciocejilla{\pgfkeysvalueof{/diagrama/iniciocejilla}}
            \def\finalcejilla{\pgfkeysvalueof{/diagrama/finalcejilla}}
            \def\numerotrastes{\pgfkeysvalueof{/diagrama/numerotrastes}}

            %Grosor de líneas
            \def\lineas{0.6pt}

            \begin{tikzpicture}
                % %Título
                \node at (11.5,-2.4) {\Huge \bfseries \textcolor{chordcolor}{%
                    \notaavalor{\nombre}%
                    \valoranota{\value{notaavalor}}%
                    \variacion%
                    }
                };%

                \begin{scope}[rotate=-90]
                    % Rectángulo externo
                    \draw[line width=\lineas] (3.75,13.5) rectangle (\ancho,9.75);

                    % Líneas horizontales
                    \draw[line width=\lineas] (3.75,12.75) -- (\ancho,12.75);
                    \draw[line width=\lineas] (3.75,12) -- (\ancho,12);
                    \draw[line width=\lineas] (3.75,11.25) -- (\ancho,11.25);
                    \draw[line width=\lineas] (3.75,10.5) -- (\ancho,10.5);

                    % Líneas verticales
                    \foreach \d in {1,2,...,\numerotrastes}{%
                        \draw[line width=\lineas] (3.75+\d*1.75,13.5) -- (3.75+\d*1.75,9.75);%
                        \pgfmathsetmacro{\suma}{int(\trasteinicial + \d - 1)}%
                        \node at (2.75+1.75*\d,9.25) {\Large \suma};%
                    }

                    % Círculos

                    % Primera cuerda
                    \ifstrequal{\primera}{0}{
                        \draw[line width=\lineas] (3.3,13.5) circle (4pt);%
                    }{%
                        \ifstrequal{\primera}{x}{
                            \node at (3.3,13.5) {\Huge {\texttimes}};%
                        }{%
                        \pgfmathsetmacro{\pos}{int(\primera-\trasteinicial+1)}%
                            \filldraw[line width=\lineas] (2.75+1.75*\pos,13.5) circle (6pt);%
                        }
                    }

                    % Segunda cuerda
                    \ifstrequal{\segunda}{0}{%
                        \draw[line width=\lineas] (3.3,12.75) circle (4pt);%
                    }{%
                        \ifstrequal{\segunda}{x}{%
                            \node at (3.3,12.75) {\Huge {\texttimes}};%
                        }{%
                        \pgfmathsetmacro{\pos}{int(\segunda-\trasteinicial+1)}%
                            \filldraw[line width=\lineas] (2.75+1.75*\pos,12.75) circle (6pt);%
                        }%
                    }

                    % Tercera cuerda
                    \ifstrequal{\tercera}{0}{%
                        \draw[line width=\lineas] (3.3,12) circle (4pt);%
                    }{%
                        \ifstrequal{\tercera}{x}{
                            \node at (3.3,12) {\Huge {\texttimes}};%
                        }{%
                            \pgfmathsetmacro{\pos}{int(\tercera-\trasteinicial+1)}%
                            \filldraw[line width=\lineas] (2.75+1.75*\pos,12) circle (6pt);%
                        }%
                    }

                    % Cuarta cuerda
                    \ifstrequal{\cuarta}{0}{%
                        \draw[line width=\lineas] (3.3,11.25) circle (4pt);%
                    }{%
                        \ifstrequal{\cuarta}{x}{%
                            \node at (3.3,11.25) {\Huge {\texttimes}};%
                        }{%
                            \pgfmathsetmacro{\pos}{int(\cuarta-\trasteinicial+1)}%
                            \filldraw[line width=\lineas] (2.75+1.75*\pos,11.25) circle (6pt);%
                        }%
                    }

                    % Quinta cuerda
                    \ifstrequal{\quinta}{0}{%
                        \draw[line width=\lineas] (3.3,10.5) circle (4pt);%
                    }{%
                        \ifstrequal{\quinta}{x}{%
                            \node at (3.3,10.5) {\Huge {\texttimes}};%
                        }{%
                            \pgfmathsetmacro{\pos}{int(\quinta-\trasteinicial+1)}%
                            \filldraw[line width=\lineas] (2.75+1.75*\pos,10.5) circle (6pt);%
                        }%
                    }

                    % Sexta cuerda
                    \ifstrequal{\sexta}{0}{%
                        \draw[line width=\lineas] (3.3,9.75) circle (4pt);%
                    }{%
                        \ifstrequal{\sexta}{x}{%
                            \node at (3.3,9.75) {\Huge {\texttimes}};%
                        }{%
                        \pgfmathsetmacro{\pos}{int(\sexta-\trasteinicial+1)}%
                            \filldraw[line width=\lineas] (2.75+1.75*\pos,9.75) circle (6pt);%
                        }%
                    }
                    
                    % Cejilla
                    \ifstrequal{\cejilla}{0}{}{%
                        \pgfmathsetmacro{\pos}{int(\cejilla-\trasteinicial+1)}%
                        \filldraw[fill=black, rounded corners=6] (2.75+1.75*\pos+0.22,-0.75*\finalcejilla+14.45) rectangle (2.75+1.75*\pos-0.22,-0.75*\iniciocejilla+14.05);
                    }

                    % Cejuela
                    \ifstrequal{\trasteinicial}{1}{%
                        \filldraw[fill=white, rounded corners=3] (3.875,13.55) rectangle (3.625,9.7);
                    }{}
                \end{scope}
            \end{tikzpicture}%
        }
    \end{figure}
}

% Parámetros por defecto para diagrama con dedos
\pgfkeys{
    /ddiagrama/.is family, /ddiagrama,
    trasteinicial/.initial=1,
    nombre/.initial=Name,
    variacion/.initial=,
    numerotrastes/.initial=3,
    cejilla/.initial=0,
    iniciocejilla/.initial=6,
    finalcejilla/.initial=1,
    height/.initial=3,
    fprimera/.initial=0,
    fsegunda/.initial=0,
    ftercera/.initial=0,
    fcuarta/.initial=0,
    fquinta/.initial=0,
    fsexta/.initial=0,
}

% Diagrama de acordes en horizontal con dedos
\newcommand{\hddiagram}[7][]{%

    % Procesar las claves
    \pgfkeys{/ddiagrama, #1}%
    
    \begin{figure}[H]%

        % Hiperlinks
        \captionsetup{labelformat=empty} % Elimina la etiqueta de la leyenda
        \caption{}%
        \def\nombre{\pgfkeysvalueof{/ddiagrama/nombre}\pgfkeysvalueof{/ddiagrama/variacion}}%
        \label{fig:\nombre}%

        \centering%

        % Ajuste tamaño figura
        \resizebox{!}{\pgfkeysvalueof{/ddiagrama/height}cm}{%  
            \def\ancho{3.75+\numerotrastes*1.75} % 9.75 - 8.25
            \def\nombre{\pgfkeysvalueof{/ddiagrama/nombre}}%
            \def\variacion{\pgfkeysvalueof{/ddiagrama/variacion}}%
            \def\primera{#2}%
            \def\segunda{#3}%
            \def\tercera{#4}%
            \def\cuarta{#5}%
            \def\quinta{#6}%
            \def\sexta{#7}%
            \def\cejilla{\pgfkeysvalueof{/ddiagrama/cejilla}}%
            \def\iniciocejilla{\pgfkeysvalueof{/ddiagrama/iniciocejilla}}%
            \def\finalcejilla{\pgfkeysvalueof{/ddiagrama/finalcejilla}}%
            \def\numerotrastes{\pgfkeysvalueof{/ddiagrama/numerotrastes}}%
            \def\trasteinicial{\pgfkeysvalueof{/ddiagrama/trasteinicial}}%
            %
            \def\fprimera{\pgfkeysvalueof{/ddiagrama/fprimera}}%
            \def\fsegunda{\pgfkeysvalueof{/ddiagrama/fsegunda}}%
            \def\ftercera{\pgfkeysvalueof{/ddiagrama/ftercera}}%
            \def\fcuarta{\pgfkeysvalueof{/ddiagrama/fcuarta}}%
            \def\fquinta{\pgfkeysvalueof{/ddiagrama/fquinta}}%
            \def\fsexta{\pgfkeysvalueof{/ddiagrama/fsexta}}%
            
            % Tamaño puntos
            \def\radio{7.5pt}%
            
            % Tamaño números
            \def\tamnum{14pt}%
            
            % Grosor de líneas
            \def\lineas{0.6pt}%

            % Grosor de cejilla
            \def\anchocejilla{0.27}
            
            \begin{tikzpicture}%
                % %Título
                \pgfmathsetmacro{\pos}{\ancho+3.75}%
                \node at (\ancho*0.5,14.5) {\Huge\bfseries\textcolor{chordcolor}{%(6.375,14.5)
                    \notaavalor{\nombre}%
                    \valoranota{\value{notaavalor}}%
                    \variacion%
                    }%
                };%

                % Rectángulo externo
                \draw[line width=\lineas] (3.75,13.5) rectangle (\ancho,9.75);%

                % Líneas horizontales
                \draw[line width=\lineas] (3.75,12.75) -- (\ancho,12.75);%
                \draw[line width=\lineas] (3.75,12) -- (\ancho,12);%
                \draw[line width=\lineas] (3.75,11.25) -- (\ancho,11.25);%
                \draw[line width=\lineas] (3.75,10.5) -- (\ancho,10.5);%

                % Líneas verticales
                \foreach \d in {1,2,...,\numerotrastes}{%
                    \draw[line width=\lineas] (3.75+\d*1.75,13.5) -- (3.75+\d*1.75,9.75);%
                    \pgfmathsetmacro{\suma}{int(\trasteinicial + \d - 1)}%
                    \node at (2.875+1.75*\d,9.25) {\Large \suma};%
                }

                % Cejilla
                \ifstrequal{\cejilla}{0}{}{%
                    \pgfmathsetmacro{\pos}{int(\cejilla-\trasteinicial+1)}%
                    \filldraw[fill=black, rounded corners=\radio] (2.875+1.75*\pos+\anchocejilla,-0.75*\finalcejilla+14.45) rectangle (2.875+1.75*\pos-\anchocejilla,-0.75*\iniciocejilla+14.05);%
                }%
                
                % Círculos

                % Primera cuerda
                \ifstrequal{\primera}{0}{%
                    \draw[line width=\lineas] (3.3,13.5) circle (4pt);%
                }{%
                    \ifstrequal{\primera}{x}{%
                        \node at (3.3,13.5) {\Huge {\texttimes}};%
                    }{%
                    \pgfmathsetmacro{\pos}{int(\primera-\trasteinicial+1)}%
                        \filldraw[line width=\lineas] (2.875+1.75*\pos,13.5) circle (\radio);%
                        \ifstrequal{\fprimera}{0}{}{%
                            \node at (2.875+1.75*\pos,13.5) {\fontsize{\tamnum}{\radio}\selectfont\textcolor{white}{\fprimera}};%
                        }%
                    }%
                }%

                % Segunda cuerda
                \ifstrequal{\segunda}{0}{%
                    \draw[line width=\lineas] (3.3,12.75) circle (4pt);%
                }{%
                    \ifstrequal{\segunda}{x}{%
                        \node at (3.3,12.75) {\Huge {\texttimes}};%
                    }{%
                    \pgfmathsetmacro{\pos}{int(\segunda-\trasteinicial+1)}%
                        \filldraw[line width=\lineas] (2.875+1.75*\pos,12.75) circle (\radio);%
                        \ifstrequal{\fsegunda}{0}{}{%
                            \node at (2.875+1.75*\pos,12.75) {\fontsize{\tamnum}{\radio}\selectfont\textcolor{white}{\fsegunda}};%
                        }%
                    }%
                }%

                % Tercera cuerda
                \ifstrequal{\tercera}{0}{%
                    \draw[line width=\lineas] (3.3,12) circle (4pt);%
                }{%
                    \ifstrequal{\tercera}{x}{
                        \node at (3.3,12) {\Huge {\texttimes}};%
                    }{%
                        \pgfmathsetmacro{\pos}{int(\tercera-\trasteinicial+1)}%
                        \filldraw[line width=\lineas] (2.875+1.75*\pos,12) circle (\radio);%
                        \ifstrequal{\ftercera}{0}{}{%
                            \node at (2.875+1.75*\pos,12) {\fontsize{\tamnum}{\radio}\selectfont\textcolor{white}{\ftercera}};%
                        }%
                    }%
                }%

                % Cuarta cuerda
                \ifstrequal{\cuarta}{0}{%
                    \draw[line width=\lineas] (3.3,11.25) circle (4pt);%
                }{%
                    \ifstrequal{\cuarta}{x}{%
                        \node at (3.3,11.25) {\Huge {\texttimes}};%
                    }{%
                        \pgfmathsetmacro{\pos}{int(\cuarta-\trasteinicial+1)}%
                        \filldraw[line width=\lineas] (2.875+1.75*\pos,11.25) circle (\radio);%
                        \ifstrequal{\fcuarta}{0}{}{%
                            \node at (2.875+1.75*\pos,11.25) {\fontsize{\tamnum}{\radio}\selectfont\textcolor{white}{\fcuarta}};
                        }%
                    }%
                }%

                % Quinta cuerda
                \ifstrequal{\quinta}{0}{%
                    \draw[line width=\lineas] (3.3,10.5) circle (4pt);%
                }{%
                    \ifstrequal{\quinta}{x}{%
                        \node at (3.3,10.5) {\Huge {\texttimes}};%
                    }{%
                        \pgfmathsetmacro{\pos}{int(\quinta-\trasteinicial+1)}%
                        \filldraw[line width=\lineas] (2.875+1.75*\pos,10.5) circle (\radio);%
                        \ifstrequal{\fquinta}{0}{}{%
                            \node at (2.875+1.75*\pos,10.5) {\fontsize{\tamnum}{\radio}\selectfont\textcolor{white}{\fquinta}};%
                        }%
                    }%
                }%

                % Sexta cuerda
                \ifstrequal{\sexta}{0}{%
                    \draw[line width=\lineas] (3.3,9.75) circle (4pt);%
                }{%
                    \ifstrequal{\sexta}{x}{%
                        \node at (3.3,9.75) {\Huge {\texttimes}};%
                    }{%
                    \pgfmathsetmacro{\pos}{int(\sexta-\trasteinicial+1)}%
                        \filldraw[line width=\lineas] (2.875+1.75*\pos,9.75) circle (\radio);%
                        \ifstrequal{\fsexta}{0}{}{%
                            \node at (2.875+1.75*\pos,9.75) {\fontsize{\tamnum}{\radio}\selectfont\textcolor{white}{\fsexta}};%
                        }%
                    }%
                }%
                
                
                

                % Cejuela
                \ifstrequal{\trasteinicial}{1}{%
                    \filldraw[fill=white, rounded corners=3] (3.875,13.55) rectangle (3.625,9.7);%
                }{}%
                
            \end{tikzpicture}%
        }%
        
    \end{figure}
}

% Diagrama de acordes en vertical con dedos
\newcommand{\vddiagram}[7][]{

    % Procesar las claves
    \pgfkeys{/ddiagrama, #1}%
    \def\anchoaux{\pgfkeysvalueof{/ddiagrama/height}}
    \pgfmathsetmacro{\ancho}{\anchoaux*1.16}%

    \begin{figure}[H]
        % Hiperlinks
        \captionsetup{labelformat=empty} % Elimina la etiqueta de la leyenda
        \caption{}
        \def\nombre{\pgfkeysvalueof{/ddiagrama/nombre}\pgfkeysvalueof{/ddiagrama/variacion}}
        \label{fig:\nombre}

        \centering

        % Ajuste tamaño figura
        \resizebox{!}{\ancho cm}{%
        \def\ancho{3.75+\numerotrastes*1.75} % 9.75 - 8.25
        \def\nombre{\pgfkeysvalueof{/ddiagrama/nombre}}
        \def\variacion{\pgfkeysvalueof{/ddiagrama/variacion}}
        \def\primera{#2}
        \def\segunda{#3}
        \def\tercera{#4}
        \def\cuarta{#5}
        \def\quinta{#6}
        \def\sexta{#7}
        \def\cejilla{\pgfkeysvalueof{/ddiagrama/cejilla}}
        \def\iniciocejilla{\pgfkeysvalueof{/ddiagrama/iniciocejilla}}
        \def\finalcejilla{\pgfkeysvalueof{/ddiagrama/finalcejilla}}
        \def\numerotrastes{\pgfkeysvalueof{/ddiagrama/numerotrastes}}
        \def\trasteinicial{\pgfkeysvalueof{/ddiagrama/trasteinicial}}

        \def\fprimera{\pgfkeysvalueof{/ddiagrama/fprimera}}
        \def\fsegunda{\pgfkeysvalueof{/ddiagrama/fsegunda}}
        \def\ftercera{\pgfkeysvalueof{/ddiagrama/ftercera}}
        \def\fcuarta{\pgfkeysvalueof{/ddiagrama/fcuarta}}
        \def\fquinta{\pgfkeysvalueof{/ddiagrama/fquinta}}
        \def\fsexta{\pgfkeysvalueof{/ddiagrama/fsexta}}

        % Tamaño puntos
        \def\radio{6pt}

        % Tamaño números
        \def\tamnum{10pt}

        % Grosor de líneas
        \def\lineas{0.6pt}

            \begin{tikzpicture}
                % %Título
                \node at (11.5,-2.4) {\Huge \bfseries \textcolor{chordcolor}{%
                    \notaavalor{\nombre}%
                    \valoranota{\value{notaavalor}}%
                    \variacion%
                    }
                };%

                \begin{scope}[rotate=-90]
                    % Rectángulo externo
                    \draw[line width=\lineas] (3.75,13.5) rectangle (\ancho,9.75);

                    % Líneas horizontales
                    \draw[line width=\lineas] (3.75,12.75) -- (\ancho,12.75);
                    \draw[line width=\lineas] (3.75,12) -- (\ancho,12);
                    \draw[line width=\lineas] (3.75,11.25) -- (\ancho,11.25);
                    \draw[line width=\lineas] (3.75,10.5) -- (\ancho,10.5);

                    % Líneas verticales
                    \foreach \d in {1,2,...,\numerotrastes}{%
                        \draw[line width=\lineas] (3.75+\d*1.75,13.5) -- (3.75+\d*1.75,9.75);%
                        \pgfmathsetmacro{\suma}{int(\trasteinicial + \d - 1)}%
                        \node at (2.75+1.75*\d,9.25) {\Large \suma};%
                    }

                    % Cejilla
                    \ifstrequal{\cejilla}{0}{}{%
                        \pgfmathsetmacro{\pos}{int(\cejilla-\trasteinicial+1)}%
                        \filldraw[fill=black, rounded corners=6] (2.875+1.75*\pos+0.22,-0.75*\finalcejilla+14.45) rectangle (2.85+1.75*\pos-0.22,-0.75*\iniciocejilla+14.05);
                    }

                    % Círculos

                    % Primera cuerda
                    \ifstrequal{\primera}{0}{
                        \draw[line width=\lineas] (3.3,13.5) circle (4pt);%
                    }{%
                        \ifstrequal{\primera}{x}{
                            \node at (3.3,13.5) {\Huge {\texttimes}};%
                        }{%
                        \pgfmathsetmacro{\pos}{int(\primera-\trasteinicial+1)}%
                            \filldraw[line width=\lineas] (2.875+1.75*\pos,13.5) circle (\radio);%
                            \ifstrequal{\fprimera}{0}{}{%
                                \node at (2.85+1.75*\pos,13.5) {\fontsize{\tamnum}{\radio}\selectfont\textcolor{white}{\fprimera}};
                            }
                        }
                    }

                    % Segunda cuerda
                    \ifstrequal{\segunda}{0}{%
                        \draw[line width=\lineas] (3.3,12.75) circle (4pt);%
                    }{%
                        \ifstrequal{\segunda}{x}{%
                            \node at (3.3,12.75) {\Huge {\texttimes}};%
                        }{%
                        \pgfmathsetmacro{\pos}{int(\segunda-\trasteinicial+1)}%
                            \filldraw[line width=\lineas] (2.875+1.75*\pos,12.75) circle (\radio);%
                            \ifstrequal{\fsegunda}{0}{}{%
                                \node at (2.85+1.75*\pos,12.75) {\fontsize{\tamnum}{\radio}\selectfont\textcolor{white}{\fsegunda}};
                            }
                        }%
                    }

                    % Tercera cuerda
                    \ifstrequal{\tercera}{0}{%
                        \draw[line width=\lineas] (3.3,12) circle (4pt);%
                    }{%
                        \ifstrequal{\tercera}{x}{
                            \node at (3.3,12) {\Huge {\texttimes}};%
                        }{%
                            \pgfmathsetmacro{\pos}{int(\tercera-\trasteinicial+1)}%
                            \filldraw[line width=\lineas] (2.875+1.75*\pos,12) circle (\radio);%
                            \ifstrequal{\ftercera}{0}{}{%
                                \node at (2.85+1.75*\pos,12) {\fontsize{\tamnum}{\radio}\selectfont\textcolor{white}{\ftercera}};
                            }
                        }%
                    }

                    % Cuarta cuerda
                    \ifstrequal{\cuarta}{0}{%
                        \draw[line width=\lineas] (3.3,11.25) circle (4pt);%
                    }{%
                        \ifstrequal{\cuarta}{x}{%
                            \node at (3.3,11.25) {\Huge {\texttimes}};%
                        }{%
                            \pgfmathsetmacro{\pos}{int(\cuarta-\trasteinicial+1)}%
                            \filldraw[line width=\lineas] (2.875+1.75*\pos,11.25) circle (\radio);%
                            \ifstrequal{\fcuarta}{0}{}{%
                                \node at (2.85+1.75*\pos,11.25) {\fontsize{\tamnum}{\radio}\selectfont\textcolor{white}{\fcuarta}};
                            }
                        }%
                    }

                    % Quinta cuerda
                    \ifstrequal{\quinta}{0}{%
                        \draw[line width=\lineas] (3.3,10.5) circle (4pt);%
                    }{%
                        \ifstrequal{\quinta}{x}{%
                            \node at (3.3,10.5) {\Huge {\texttimes}};%
                        }{%
                            \pgfmathsetmacro{\pos}{int(\quinta-\trasteinicial+1)}%
                            \filldraw[line width=\lineas] (2.875+1.75*\pos,10.5) circle (\radio);%
                            \ifstrequal{\fquinta}{0}{}{%
                                \node at (2.85+1.75*\pos,10.5) {\fontsize{\tamnum}{\radio}\selectfont\textcolor{white}{\fquinta}};
                            }
                        }%
                    }

                    % Sexta cuerda
                    \ifstrequal{\sexta}{0}{%
                        \draw[line width=\lineas] (3.3,9.75) circle (4pt);%
                    }{%
                        \ifstrequal{\sexta}{x}{%
                            \node at (3.3,9.75) {\Huge {\texttimes}};%
                        }{%
                        \pgfmathsetmacro{\pos}{int(\sexta-\trasteinicial+1)}%
                            \filldraw[line width=\lineas] (2.875+1.75*\pos,9.75) circle (\radio);%
                            \ifstrequal{\fsexta}{0}{}{%
                                \node at (2.85+1.75*\pos,9.75) {\fontsize{\tamnum}{\radio}\selectfont\textcolor{white}{\fsexta}};
                            }
                        }%
                    }
                    
                    

                    % Cejuela
                    \ifstrequal{\trasteinicial}{1}{%
                        \filldraw[fill=white, rounded corners=3] (3.875,13.55) rectangle (3.625,9.7);
                    }{}
                \end{scope}
            \end{tikzpicture}%
        }
    \end{figure}
}

% Parámetros por defecto para portada
\pgfkeys{
    /portada/.is family, /portada,
    titulo/.initial=,
    autor/.initial=,
    textcolor/.initial=white,
    subtitulo/.initial=,
    image/.initial=imagen_portada.jpeg,
    boxcolor/.initial=,
}

% Portada
\newcommand{\portada}[1][]{ % Argumento opcional
\pgfkeys{/portada, #1}
    \thispagestyle{empty}
    \def\titulo{\pgfkeysvalueof{/portada/titulo}}
    \def\subtitulo{\pgfkeysvalueof{/portada/subtitulo}}
    \def\autor{\pgfkeysvalueof{/portada/autor}}
    \def\fontcolor{\pgfkeysvalueof{/portada/textcolor}}
    \def\imagen{\dirimages\pgfkeysvalueof{/portada/image}}

    \pgfkeys{/portada, boxcolor/.get=\parametrovalor}%
    \ifx\parametrovalor\empty%
        \colorlet{caja}{chordcolor}
    \else%
        \definecolor{caja}{HTML}{\parametrovalor}
    \fi%

    % Figuras
    \begin{tikzpicture}[remember picture, overlay]

        % Fondo
        \node[inner sep=0pt, anchor=south west] at (current page.south west) {\includegraphics[width=\paperwidth, height=\paperheight]{\imagen}};

        % Rectángulo
        % \draw [chordcolor,fill={chordcolor}, rotate=20, opacity=0.9, rounded corners=1cm] (-3,-6.6) rectangle (19.6,-18.6);
        \draw [caja,fill={caja}, rotate=20, opacity=0.9, rounded corners=1cm] (-3,-11.6) rectangle (20.6,-23.6);

        % Título
        \node at (8.79,-14) {\centering \fontsize{60}{14}\selectfont \bfseries \textcolor{\fontcolor}{\titulo}};

        % Subtítulo
        \node at (8.79,-16) {\centering \fontsize{30}{14}\selectfont \textcolor{\fontcolor}{\subtitulo}};

        % Autor
        \node[rotate=20] at (16.75,-18.3) {\centering \fontsize{15}{14}\selectfont \textcolor{\fontcolor}{\autor}};

        % Para calcular el centro
        % \draw (current page.north east) -- (current page.south west);
        % \draw (current page.north west) -- (current page.south east);

        %\fill (8.8,-11.6) circle (10pt); % Dibuja un punto en la coordenada (0,0) con un radio de 2pt

    \end{tikzpicture}
    \newpage
    
}

% Intro
\newcommand{\intro}[1]{\textbf{Intro:} \chordstyle{#1}}

% Licencia
\newcommand{\licence}{%
    \thispagestyle{empty}               % Sin encabezado ni pie de página
    \begin{tikzpicture}[remember picture, overlay]
        \node[anchor=south west, inner sep=3cm] at (current page.south west) {
            \begin{minipage}{0.5\paperwidth}
                \href{https://creativecommons.org/licenses/by-nc-nd/4.0/deed.es}{
                    \includegraphics[height=2cm]{\dirimages/Licencia.png}
                }\vspace{1cm}\\
                Esta obra está bajo una
                \href{https://creativecommons.org/licenses/by-nc-nd/4.0/deed.es}{
                    Licencia Creative Commons Atribución-NoComercial-SinDerivadas 4.0 Internacional (CC BY-NC-ND 4.0).
                }\\

                Eres libre de compartir y redistribuir el contenido de esta obra en cualquier medio o formato, siempre y cuando des el crédito adecuado a los autores originales y no persigas fines comerciales. 
            \end{minipage}
        };
    \end{tikzpicture}
}