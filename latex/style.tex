% Paquete para crear comandos más personalizados
\usepackage{xparse}

% Paquete para usar varias columnas
\usepackage{multicol}

% Paquete para colores personalizados
\usepackage{xcolor}

% Paquete para la imagen de la portada
\usepackage{graphicx}

% Paquete para parámetros adicionales
\usepackage{pgfkeys}

% Paquete para figuras
\usepackage{tikz}

% Paquete para colocar los acordes encima de las palabras
\usepackage{stackengine}
\usepackage{mathtools}
\stackMath

% Paquete para evitar el salto de página al usar twocolumn
\usepackage{flushend}

% Paquete para enlaces del índice
\usepackage{hyperref}

% Paquete para personalizar el interlineado
\usepackage{setspace}

% Tipografía
\usepackage[defaultfam,tabular,lining]{montserrat} %% Option 'defaultfam'
%% only if the base font of the document is to be sans serif
\usepackage[T1]{fontenc}
\renewcommand*\oldstylenums[1]{{\fontfamily{Montserrat-TOsF}\selectfont #1}}

% Directorio de imágenes
\graphicspath{/home/jesusmuve/Documentos/VSCode/LaTeX/documentos/songs/}
\def\dirimages{/home/jesusmuve/Documentos/VSCode/LaTeX/documentos/songs/images/}

% Ajustar las 2 columnas
\setlength{\columnsep}{40pt}   % Espacio entre columnas
\setlength{\columnseprule}{0pt}  % Grosor de la línea divisoria


% Numeración de canciones
\newcounter{contador_canciones}

% Numeración de secciones
\newcounter{contador_secciones}

% Cambiar el título del índice
\renewcommand{\contentsname}{Índice}

% Definir cómo va a aparecer una seccion
\newcommand{\seccion}[1]{
    \stepcounter{contador_secciones}%
    \addcontentsline{toc}{section}{\arabic{contador_secciones}. #1}%
    \textbf{\Huge{#1}}\\
    \vspace*{0.25cm}
}

% Establecer longitud de la sangría a cero
\setlength{\parindent}{0pt}

% Distancia entre párrafos
\newcommand{\jump}{\vspace*{0.5cm}}

% Interlineado de las canciones
\def\interlineadocanciones{1.75}

% Estribillo
\newenvironment{chorus}{%
    \bfseries
}{}

% Color para los acordes
\definecolor{chordcolor}{RGB}{255,87,51}

% Cambio de columna
\newcommand{\cambiocolumna}{%
    \vspace*{\fill}
    \columnbreak

}

\newcommand{\huecotitulo}{%
    \phantom{\large \textbf{\arabic{contador_canciones}. \titulo}}\\
    \phantom{\phantom{\large \textbf{\arabic{contador_canciones}. }}\small \textit{\subtitulo}}
    \vspace*{-0.5cm}\\
}

% Formato para los acordes
\newcommand{\chordstyle}[1]{%
    \fontsize{8pt}{12pt}\textcolor{chordcolor}{\selectfont\bfseries #1}%
}

% Definición de acorde
\newcommand{\chord}[3]{%
    \renewcommand\useanchorwidth{T}%
    \stackon{\text{#3\vphantom{é}}}{\text{\chordstyle{#1#2}}}%
}


\NewDocumentEnvironment{cancion}{O{} O{}}{%
    \stepcounter{contador_canciones}%
    \addcontentsline{toc}{subsection}{\arabic{contador_canciones}. #1\ (\textit{#2})}%
    \large \textbf{\arabic{contador_canciones}. #1}\\ % Título
    \def\titulo{#1}%
    \phantom{\large \textbf{\arabic{contador_canciones}. }}\small \textit{#2}\vspace*{-0.5cm}\\ % Subtítulo
    \def\subtitulo{#2}%
    \begin{spacing}{\interlineadocanciones}%
}%
{%
    \end{spacing}%
}

\NewDocumentCommand{\doscolumnas}{m O{10cm}}{%
    \vbox to #2{%
        \begin{multicols}{2}%
            #1
        \end{multicols}
        \vfil%
    }
}

% Parámetros por defecto para portada
\pgfkeys{
    /portada/.is family, /portada,
    titulo/.initial=,
    autor/.initial=,
    textcolor/.initial=white,
    subtitulo/.initial=,
    image/.initial=imagen_portada.jpeg,
    boxcolor/.initial=,
}

% Portada
\newcommand{\portada}[1][]{ % Argumento opcional
\pgfkeys{/portada, #1}
    \thispagestyle{empty}
    \def\titulo{\pgfkeysvalueof{/portada/titulo}}
    \def\subtitulo{\pgfkeysvalueof{/portada/subtitulo}}
    \def\autor{\pgfkeysvalueof{/portada/autor}}
    \def\fontcolor{\pgfkeysvalueof{/portada/textcolor}}
    \def\imagen{\dirimages\pgfkeysvalueof{/portada/image}}

    \pgfkeys{/portada, boxcolor/.get=\parametrovalor}%
    \ifx\parametrovalor\empty%
        \colorlet{caja}{chordcolor}
    \else%
        \definecolor{caja}{HTML}{\parametrovalor}
    \fi%

    % Figuras
    \begin{tikzpicture}[remember picture, overlay]

        % Fondo
        \node[inner sep=0pt, anchor=south west] at (current page.south west) {\includegraphics[width=\paperwidth, height=\paperheight]{\imagen}};

        % Rectángulo
        % \draw [chordcolor,fill={chordcolor}, rotate=20, opacity=0.9, rounded corners=1cm] (-3,-6.6) rectangle (19.6,-18.6);
        \draw [caja,fill={caja}, rotate=20, opacity=0.9, rounded corners=1cm] (-3,-11.6) rectangle (20.6,-23.6);

        % Título
        \node at (8.79,-14) {\centering \fontsize{60}{14}\selectfont \bfseries \textcolor{\fontcolor}{\titulo}};

        % Subtítulo
        \node at (8.79,-16) {\centering \fontsize{30}{14}\selectfont \textcolor{\fontcolor}{\subtitulo}};

        % Autor
        \node[rotate=20] at (16.75,-18.3) {\centering \fontsize{15}{14}\selectfont \textcolor{\fontcolor}{\autor}};

        % Para calcular el centro
        % \draw (current page.north east) -- (current page.south west);
        % \draw (current page.north west) -- (current page.south east);

        %\fill (8.8,-11.6) circle (10pt); % Dibuja un punto en la coordenada (0,0) con un radio de 2pt

    \end{tikzpicture}
    \newpage
    
}

% Intro
\newcommand{\intro}[1]{\textbf{Intro:} \chordstyle{#1}}