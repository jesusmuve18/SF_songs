\usepackage{fontspec} % Para cambiar la fuente

\usepackage{multicol} % Para varias columnas

\usepackage{luacode} % Para añadir código

\usepackage{imakeidx} % Para manejar el índice
\makeindex

\setmainfont{DMSans-Regular}[
    Path=./tipografia/DM_Sans/static/,
    BoldFont=DMSans-Bold.ttf,
    ItalicFont=DMSans-Italic,
    BoldItalicFont=DMSans-BoldItalic.ttf]

% Definir una nueva familia para la variante Light
\newfontfamily\lightfont{DMSans-Light.ttf}[
    Path=./tipografia/DM_Sans/static/]
  
% Crear un comando para usar la fuente Light
\newcommand{\textlight}[1]{{\lightfont #1}}

% Código Lua para leer el índice
\begin{luacode*}
    function readIndexFile(filename)
      local index_entries = {}
      local file = io.open(filename, "r")
      if file then
        for line in file:lines() do
          local entry = line:match("\\indexentry%{([^}]*)%}")
          if entry then
            table.insert(index_entries, entry)
          end
        end
        file:close()
      end
      return index_entries
    end
    
    index_entries = readIndexFile("test.idx")  -- Ajusta el nombre según tu archivo
\end{luacode*}
    

