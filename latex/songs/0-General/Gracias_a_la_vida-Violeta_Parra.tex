\begin{cancion}[Gracias a la vida][Violeta Parra]%
	\chord{la}{m}{Gra}cias a la \chord{re}{m}{vid}a \\
	\chord{Mi}{}{qu}e me ha dado \chord{la}{m}{tan}to.\\
	Me dio los lu\chord{Sol}{}{cer}os que, \\
	cuando los \chord{Do}{7}{abr}o,\\
	perfecto dis\chord{sol}{m}{ting}o lo negro y lo \chord{Fa}{}{bl}anco\\
	y en el alto \chord{re}{m}{cie}lo, \\
	su \chord{Mi}{}{fo}ndo estrel\chord{la}{m}{lad}o\\
	y en las mult\chord{re}{m}{itu}des\\
	al \chord{Mi}{}{ho}mbre que yo \chord{la}{m}{amo}.\\
	\jump\\
	\chord{la}{m}{Gra}cias a la \chord{re}{m}{vid}a \\
	\chord{Mi}{}{qu}e me ha dado \chord{la}{m}{tan}to.\\
	Me ha dado \chord{Sol}{}{el }sonido, \\
	y el abecedar\chord{Do}{7}{io,}\\
	con él, las p\chord{sol}{m}{alab}ras que pienso \\
	y\chord{Fa}{}{ d}eclaro:\\
	padre, amig\chord{re}{m}{o, }herma\chord{Mi}{}{no} y luz alum\chord{la}{m}{bra}ndo\\
	la ruta del al\chord{re}{m}{ma }del qu\chord{Mi}{}{e }estoy aman\chord{la}{m}{do.}\\
	\jump\\
	\chord{la}{m}{Gra}cias a la \chord{re}{m}{vid}a \\
	\chord{Mi}{}{qu}e me ha dado \chord{la}{m}{tan}to.\\
	Me ha dado \chord{Sol}{}{el }oído\\
	y en todo su anc\chord{Do}{7}{ho }\\
	braman noch\chord{sol}{m}{e y }día ríos y can\chord{Fa}{}{ar}ios,\\
	martillos, \chord{re}{m}{tur}binas, \\
	l\chord{Mi}{}{ad}ridos, chu\chord{la}{m}{bas}cos,\\
	y la voz tan t\chord{re}{m}{ier}na de \chord{Mi}{}{mi} buen amad\chord{la}{m}{o. }\\
	\jump\\
	\chord{la}{m}{Gra}cias a la \chord{re}{m}{vid}a \\
	\chord{Mi}{}{qu}e me ha dado \chord{la}{m}{tan}to.\\
	Me ha dado \chord{Sol}{}{la }marcha \\
	de mis pies cansa\chord{Do}{7}{dos},\\
	con ellos an\chord{sol}{m}{duve} senderos y c\chord{Fa}{}{ha}rcos,\\
	playas y des\chord{re}{m}{ier}tos, m\chord{Mi}{}{on}tañas y \chord{la}{m}{lla}nos,\\
	y la casa tu\chord{re}{m}{ya,} tu calle\chord{Mi}{}{, }tu patio.\\
	\jump\\
	\chord{la}{m}{Gra}cias a la \chord{re}{m}{vid}a \\
	\chord{Mi}{}{qu}e me ha dado \chord{la}{m}{tan}to.\\
	Me ha dado \chord{Sol}{}{la }dicha, \\
	me ha dado el \chord{Do}{7}{lla}nto,\\
	así yo disti\chord{sol}{m}{ngo }dicha de quebr\chord{Fa}{}{an}to,\\
	los dos mate\chord{re}{m}{ria}les qu\chord{Mi}{}{e }forman mi\chord{la}{m}{ ca}nto\\
	y el canto de \chord{re}{m}{ust}edes, \\
	\chord{Mi}{}{qu}e es mi prop\chord{la}{m}{io }canto.\\
\end{cancion}%
