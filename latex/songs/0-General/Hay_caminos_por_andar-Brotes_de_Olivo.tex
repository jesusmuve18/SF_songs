\begin{cancion}[Hay caminos por andar][Brotes de Olivo]jump\\
	\begin{chorus}%
		\chord{Do}{}{Tú} que andas:\chord{Fa}{}{ s}í hay c\chord{Do}{}{am}inos,  \\
		lo que hace \chord{Fa}{}{fa}lta es bus\chord{Sol}{}{car}.\\
		Cami\chord{Do}{}{na}nt\chord{Fa}{}{e,} no lo \chord{Do}{}{du}des,\chord{La}{7}{   }\\
		\chord{re}{m}{hay} ca\chord{Sol}{}{min}os por an\chord{Do}{}{da}r.\jump\\
	\end{chorus}%
	\chord{Do}{}{Ha}y caminos por an\chord{Sol}{}{dar} \\
	o que \chord{Fa}{}{po}co se andu\chord{Do}{}{vi}eron\\
	y aunque muchos se can\chord{Sol}{}{sar}on, \\
	fue an\chord{Fa}{}{da}ndo senderos v\chord{Do}{}{ie}jos.\chord{Do}{7}{   }\\
	Como el \chord{Fa}{}{pu}eblo de Israel \\
	que en su an\chord{Sol}{}{dar} por el de\chord{Do}{}{si}erto\\
	olvidó su libe\chord{Sol}{}{rta}d \\
	\chord{Fa}{}{pa}ra añorar el de\chord{Do}{}{st}ierro.\chord{Sol}{}{   }\\
	\jump
	\chord{Do}{}{Cu}ando la ilusión y e\chord{Sol}{}{l g}ozo \\
	re\chord{Fa}{}{pi}ten sus senti\chord{Do}{}{mi}entos,\\
	se está matando la\chord{Sol}{}{ vi}da\\
	que qui\chord{Fa}{}{er}e alcanzar lo e\chord{Do}{}{te}rno.\chord{Do}{7}{   }\\
	No “lo\chord{Fa}{}{ e}terno” del mañana, \\
	que se\chord{Sol}{}{rá} siempre algo in\chord{Do}{}{ci}erto\\
	sino vivir hoy, en to\chord{Sol}{}{dos}, \\
	\chord{Fa}{}{co}mo se vive en el \chord{Do}{}{ci}elo. \chord{Sol}{}{   }\\
	\jump
	\chord{Do}{}{Si} queremos un fu\chord{Sol}{}{tur}o \\
	f\chord{Fa}{}{ru}to de un presente in\chord{Do}{}{ci}erto,\\
	no podemos espe\chord{Sol}{}{rar} más \\
	que \chord{Fa}{}{du}da y descon\chord{Do}{}{ci}erto.\chord{Do}{7}{   }\\
	\chord{Fa}{}{so}lo el presente que vive \\
	para \chord{Sol}{}{hac}er realidad los \chord{Do}{}{su}eños\\
	un futuro nos \chord{Sol}{}{dar}á \\
	\chord{Fa}{}{qu}e haga presente el \chord{Do}{}{Re}ino. \chord{Sol}{}{   }\\
	\jump
	\chord{Do}{}{Ha}y caminos por an\chord{Sol}{}{dar} \\
	que al \chord{Fa}{}{cr}uzarlos son \chord{Do}{}{vi}olentos\\
	y entre temblores y \chord{Sol}{}{dud}as \\
	se re\chord{Fa}{}{co}rren muchos tre\chord{Do}{}{ch}os. \chord{Do}{7}{   }\\
	Más\chord{Fa}{}{ a}lgo interior nos dice: \\
	“En m\chord{Sol}{}{i n}ombre, re\chord{Do}{}{co}rredlo,\\
	que yo voy por de\chord{Sol}{}{lan}te, \\
	\chord{Fa}{}{Yo} soy: ¡no tengái\chord{Do}{}{s }miedo!\chord{Sol}{}{”}\\
\end{cancion}%
