% Paquete para crear comandos más personalizados
\usepackage{xparse}

% Paquete para usar varias columnas
\usepackage{multicol}

% Paquete para colores personalizados
\usepackage{xcolor}

% Paquete para la imagen de la portada
\usepackage{graphicx}

% Paquete para parámetros adicionales
\usepackage{pgfkeys}

% Paquete para figuras
\usepackage{tikz}

% Paquete para colocar los acordes encima de las palabras
\usepackage{stackengine}
\usepackage{mathtools}
\stackMath

% Paquete para evitar el salto de página al usar twocolumn
\usepackage{flushend}

% Paquete para enlaces del índice
\usepackage{hyperref}

% Paquete para personalizar el interlineado
\usepackage{setspace}

% Paquete para eliminar líneas vacías cuando hay varias líneas de acordes
\usepackage{transparent}

% Paquete para hacer comprobaciones de parámetros
\usepackage{xifthen}

% Tipografía
% \usepackage[defaultfam,tabular,lining]{montserrat} %% Option 'defaultfam'
% %% only if the base font of the document is to be sans serif
% \usepackage[T1]{fontenc}
% \renewcommand*\oldstylenums[1]{{\fontfamily{Montserrat-TOsF}\selectfont #1}}

 %para cambiar el tipo de fuente por defecto
 \renewcommand{\sfdefault}{phv}
 % helvética en los títulos
 \renewcommand{\rmfamily}{phv}

% Directorio de imágenes
\graphicspath{/home/jesusmuve/Documentos/VSCode/LaTeX/documentos/songs/}
\def\dirimages{/home/jesusmuve/Documentos/VSCode/Web/sanfran/latex/cantoral/images/}

% Ajustar las 2 columnas
\setlength{\columnsep}{40pt}   % Espacio entre columnas
\setlength{\columnseprule}{0pt}  % Grosor de la línea divisoria

% Numeración de canciones
\newcounter{contador_canciones}

% Numeración de secciones
\newcounter{contador_secciones}

% Cambiar el título del índice
\renewcommand{\contentsname}{Índice}

% Definir cómo va a aparecer una seccion
\newcommand{\seccion}[1]{
    \stepcounter{contador_secciones}%
    \setcounter{contador_canciones}{0}% El contador de canciones se reinicia en cada sección
    \addcontentsline{toc}{section}{\arabic{contador_secciones}. #1}%
    \textbf{\Huge{#1}}\\
    \vspace*{0.25cm}
    
}

% Establecer longitud de la sangría a cero
\setlength{\parindent}{0pt}

% Distancia entre párrafos
\newcommand{\jump}{\vspace*{-0.25cm}}

% Interlineado de las canciones
\def\interlineadocanciones{1.75}

% Estribillo
\newenvironment{chorus}{%
    \bfseries
}{}

% Color para los acordes
% \definecolor{chordcolor}{RGB}{255,87,51}
\definecolor{chordcolor}{RGB}{255,0,0}

% Cambio de columna
\newcommand{\siguientecolumna}{%
    \vspace*{\fill}
    \columnbreak

}

\newcommand{\huecotitulo}{%
    \phantom{\large \textbf{\arabic{contador_canciones}. \titulo}}\\
    \phantom{\phantom{\large \textbf{\arabic{contador_canciones}. }}\small \textit{\subtitulo}}
    \vspace*{-0.5cm}\\
}

% Formato para los acordes
\newcommand{\chordstyle}[1]{%
    \fontsize{8pt}{12pt}\textcolor{chordcolor}{\selectfont\bfseries #1}%
}

% Definición de acorde
% \newcommand{\chord}[3]{%
%     \renewcommand\useanchorwidth{T}%
%     \stackon{\text{#3\vphantom{(}}}{\text{\chordstyle{\transparent{1}\spaceskip=0.5cm\relax#1#2}\vphantom{(}}}%
% }
\newcommand{\chord}[3]{%
  \setbox0=\hbox{#3}% almacenar el primer argumento en una caja
  \setbox1=\hbox to \wd0{\chordstyle{#1#2\hss}}% Crear caja superior con el mismo ancho y centrado
  \leavevmode% asegurar que el contenido se renderiza en modo horizontal
  \rlap{\copy0}% renderizar el contenido base sin mover el punto de referencia
  \raisebox{2ex}{\copy1}% apilar el contenido sobre el base con ajuste vertical
}


\NewDocumentEnvironment{cancion}{O{} O{}}{%
    \stepcounter{contador_canciones}%
    \ifthenelse{\isempty{#2}}%
        {\addcontentsline{toc}{subsection}{\arabic{contador_canciones}. #1\ }} % Si no tiene autor especificado
        {\addcontentsline{toc}{subsection}{\arabic{contador_canciones}. #1\ (\textit{#2})}} % Si tiene autor especificado
    \large \textbf{\arabic{contador_canciones}. #1}\\ % Título
    \def\titulo{#1}%
    \def\subtitulo{#2}%
    \phantom{\large \textbf{\arabic{contador_canciones}. }}\small \textit{#2}%
    % \vspace{0.5cm}
    \begin{spacing}{\interlineadocanciones}%
        \obeyspaces\spaceskip=0.11cm\relax% Personaliza el ancho de los espacios en blanco
        
}%
{%
    \end{spacing}%
}

\NewDocumentCommand{\doscolumnas}{m O{10cm}}{%
    \vbox to #2{%
        \begin{multicols}{2}%
            #1
        \end{multicols}
        \vfil%
    }
}

% Parámetros por defecto para portada
\pgfkeys{
    /portada/.is family, /portada,
    titulo/.initial=,
    textcolor/.initial=black,
    subtitulo/.initial=,
    image/.initial=portada.png,
    logo/.initial=logo.png,
}

% Portada
\newcommand{\portada}[1][]{ % Argumento opcional
    \pgfkeys{/portada, #1}
    \thispagestyle{empty}
    \def\titulo{\pgfkeysvalueof{/portada/titulo}}
    \def\subtitulo{\pgfkeysvalueof{/portada/subtitulo}}
    \def\fontcolor{\pgfkeysvalueof{/portada/textcolor}}
    \def\imagen{\dirimages\pgfkeysvalueof{/portada/image}}
    \def\logo{\dirimages\pgfkeysvalueof{/portada/logo}}

    \thispagestyle{empty}
    \vspace*{0.5cm}

    \begin{center}
        \begin{tikzpicture}
            % Título
            \node[text width=15cm, inner sep=15pt, align=center] at (0,0) {
                \centering \fontsize{60}{14}\selectfont \bfseries \textcolor{\fontcolor}{\titulo}
            };

            % Subtítulo usando parbox para mejorar espaciado
            \node[text width=15cm, align=center, inner sep=30pt] at (0,-3) {
                \parbox{15cm}{
                    \centering
                    \fontsize{35}{14}\selectfont
                    {\subtitulo}
                }
            };

            % Imagen
            \node at (0,-12) {\includegraphics[width=16cm]{\imagen}};

            % Logo
            \node at (0,-20) {\includegraphics[width=2cm]{\logo}};
        \end{tikzpicture}
    \end{center}
    
    \newpage
}


% Intro
\newcommand{\intro}[1]{\textbf{Intro:} \chordstyle{#1}}