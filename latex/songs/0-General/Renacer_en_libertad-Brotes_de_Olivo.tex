\begin{cancion}[Renacer en libertad][Brotes de Olivo]%
	\begin{chorus}%
	No \chord{Sol}{}{qui}ero una libertad \\
que me hace libre frente a Ti.\\
	No \chord{mi}{m}{qui}ero quererte más\\
por imponérmelo a mí mismo.\\
	No \chord{Do}{}{qu}iero más regla que tu amor.\\
	No \chord{Re}{}{qu}iero más regla que tu voz.\\
	\end{chorus}%
	\jump\\
	Qu\chord{Sol}{}{ere}mos estrenar vivir en alegría,\\
	en \chord{mi}{m}{com}unidad, que nazca un nuevo día.\\
	En \chord{la}{m}{med}io de problemas, \\
luchas y acercamientos,\\
	cr\chord{Re}{}{ec}er hacia Ti  \\
	desde un \chord{(Do}{)}{nuev}o pens\chord{(Re}{)}{amie}nto.\\\jump\\
	\begin{chorus}%
	No \chord{Sol}{}{qui}ero una libertad \\
que me hace libre frente a Ti.\\
	No \chord{mi}{m}{qui}ero quererte más\\
por imponérmelo a mí mismo.\\
	No \chord{Do}{}{qu}iero más regla que tu amor.\\
	No \chord{Re}{}{qu}iero más regla que tu voz.\\
	\end{chorus}%
	\jump\\
	\chord{Sol}{}{Par}a edificar la casa sobre roca,\\
	fuera \chord{mi}{m}{dud}as, categorías, \\
fuera las normas.\\
	El \chord{la}{m}{Rei}no se con\chord{(Sol}{/Re}{struye }\\
	c\chord{fa\#}{/Re}{onfian}do en mi \chord{mi}{m7)}{voz, }\\
	sin\chord{Re}{}{ti}endo e\chord{(Do}{}{n l}a prueba \\
	"\chord{Si}{/Re}{no te}máis, soy\chord{La}{/Re)}{ yo". }\\\jump\\
	\begin{chorus}%
	No \chord{Sol}{}{qui}ero una libertad \\
que me hace libre frente a Ti.\\
	No \chord{mi}{m}{qui}ero quererte más\\
por imponérmelo a mí mismo.\\
	No \chord{Do}{}{qu}iero más regla que tu amor.\\
	No \chord{Re}{}{qu}iero más regla que tu voz.\\
	\end{chorus}%
	\jump\\
	\chord{Sol}{}{Viv}e la libertad\\
	mi\chord{mi}{m}{rá}ndote en los demás.\\
	\chord{la}{m}{Sie}nte que creces al servir,\\
	que los d\chord{Re}{}{em}ás renacen... \\
Tú con ellos, al fin. (**)\\\jump\\
	\begin{chorus}%
	No \chord{Sol}{}{qui}ero una libertad \\
que me hace libre frente a Ti.\\
	No \chord{mi}{m}{qui}ero quererte más\\
por imponérmelo a mí mismo.\\
	No \chord{Do}{}{qu}iero más regla que tu amor.\\
	No \chord{Re}{}{qu}iero más regla que tu voz.\\
	No quiero.\chord{Sol}{}{.. }\\
	\end{chorus}%
	\jump\\
\end{cancion}%
