\begin{cancion}[Cosas de locos][Alberto y Emilia]%
	\chord{Mi}{}{Si} pienso lo que quieres que yo \chord{La}{}{ha}ga,\\
	si tengo\chord{Mi}{}{ q}ue hacer caso \\
	a tus pa\chord{La}{}{la}bras,\\
	si quieres \chord{fa\#}{m}{que }te diga lo que \chord{Si}{}{pi}enso,\\
	que es \chord{La}{}{de} locos, que es de \chord{Mi}{}{lo}cos.\\
\jump\\
	\chord{Mi}{}{Si} quieres que me quite las ca\chord{La}{}{de}nas\\
	que me hac\chord{Mi}{}{en} sentir seguro aq\chord{La}{}{uí} abajo, \\
	es como\chord{fa\#}{m}{ si }me vaciara las v\chord{Si}{}{en}as.\\
	Tú está\chord{La}{}{s }loco, tú est\chord{Mi}{}{á}s l\chord{Mi}{7}{oco}.\\\jump\\
	\begin{chorus}%
	\chord{La}{}{Só}lo te pido \chord{Si}{}{fu}erzas para \chord{Mi}{}{ha}cer \\
	de mi debilid\chord{La}{}{ad} un \chord{Si}{}{fé}rreo venda\chord{Mi}{}{va}l\chord{Mi}{7}{.  }\\
	\chord{La}{}{De}sde el convenci\chord{Si}{}{mi}ento que tal v\chord{Mi}{}{ez}\\
	hoy todo puede s\chord{La}{}{er} de n\chord{fa\#}{m}{uevo} reali\chord{La}{}{da}d,\\
	que \chord{Si}{}{ya} estás al ll\chord{Mi}{}{eg}ar. \chord{La}{}{  } \chord{la}{m}{   }\\
	\end{chorus}%
	\jump\\
	\chord{Mi}{}{De} todas formas sé que es nec\chord{La}{}{es}ario\\
	andar \chord{Mi}{}{co}ntra corriente en esta \chord{La}{}{ti}erra\\
	y que en e\chord{fa\#}{m}{l fo}ndo merece la \chord{Si}{}{pe}na\\
	estar \chord{La}{}{lo}co, estar \chord{Mi}{}{lo}co.\\\jump\\
	\begin{chorus}%
	\chord{La}{}{Só}lo te pido \chord{Si}{}{fu}erzas para \chord{Mi}{}{ha}cer \\
	de mi debilid\chord{La}{}{ad} un \chord{Si}{}{fé}rreo venda\chord{Mi}{}{va}l\chord{Mi}{7}{.  }\\
	\chord{La}{}{De}sde el convenci\chord{Si}{}{mi}ento que tal v\chord{Mi}{}{ez}\\
	hoy todo puede s\chord{La}{}{er} de n\chord{fa\#}{m}{uevo} reali\chord{La}{}{da}d,\\
	que \chord{Si}{}{ya} estás al ll\chord{Mi}{}{eg}ar. \chord{La}{}{(*}*\chord{la}{m}{)  }\\
	\end{chorus}%
	\jump\\
\end{cancion}%
